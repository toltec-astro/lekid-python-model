\documentclass[12pt]{article}

\usepackage{graphicx}
\usepackage[margin=1in]{geometry} 
\usepackage[UKenglish]{isodate}
\usepackage{amsmath,amsthm,amssymb}
\usepackage{enumerate}
\usepackage[separate-uncertainty=true]{siunitx}
\usepackage{boldline}
\usepackage{float}
\restylefloat{table}
%\usepackage{mhchem}

\begin{document}

\title{Code equations}%replace with the appropriate homework number
\author{Yvonne Ban}%replace with your name
\cleanlookdateon
\maketitle
\tableofcontents

\newpage
\section{Empirical inputs}
\begin{table}[H]
\caption{Empirical inputs}
\begin{center}
\begin{tabular}{|c|c|c|}
\hline
Input & Value & Confidence\\\hlineB{2}
wI, width of inductor & \SI{4e-6}{m} & 100\%\\\hline
lI, length of inductor & \SI{4e-4}{m} & 100\%\\\hline
tI, thickness of inductor & \SI{1.8e-8}{m} & 100\%\\\hline
Tc, critical temperature & \SI{1.4}{K} & 100\%\\\hline
tau0, characteristic electron-phonon interaction time & \SI{4.38e-7}{s} & 0\%\\\hline
T, operating temperature & \SI{0.1}{K} & 100\%\\\hline
R\_nsp, normal sheet resistivity & 90 Ohms/square & 100\%\\\hline
lamb, wavelength of photons & \SI{1.1e-3}{m} & 100\%\\\hline
delta\_nuopt, optical bandwidth & \SI{1e10}{Hz} & 90\%\\\hline
Lg, geometric inductance & \SI{3e-9}{H} & 100\%\\\hline
C, capacitor capacitance in circuit & \SI{5e-12}{F} & 100\%\\\hline
eta\_opt, optical efficiency & 0.8 & 100\%\\\hline
eta\_pb, pair-breaking efficiency & 0.57 & 100\%\\\hline
tau\_phon\_br, time for phonon to break Cooper pair & \SI{1e-10}{s} & 0\%\\\hline
Qc, coupling quality factor & \num{5e4} & 100\%\\\hline
nqp0, quasiparticle density at 0K & \num{1e20} \si{m^{-3}} & 100\%\\\hline
P\_read, readout power & \SI{6.31e-12}{W} & 0\%\\\hline
T\_amp, amplifier noise temperature & \SI{3}{K} & 0\%\\\hline
deltav\_read, readout bandwidth & \SI{50}{Hz} & 0\%\\\hline
V\_read, readout voltage & \SI{1.776e-5}{V} & 0\%\\\hline
\end{tabular}
\end{center}
\label{tab:emp}
\end{table}

\section{Constants}
\subsection{VI, volume of inductor}

\begin{table}[H]
\caption{Inputs to VI, volume of inductor}
\begin{center}
\begin{tabular}{|c|c|}
\hline
Input & Confidence\\\hlineB{2}
wI, width of inductor & 100\%\\\hline
lI, length of inductor & 100\%\\\hline
tI, thickness of inductor & 100\%\\\hline
\end{tabular}
\end{center}
%\label{tab:VI}
\end{table}

\begin{align*}
V_I &= w_I l_I t_I
\end{align*}

\subsection{nsq, number of squares of inductor}

\begin{table}[H]
\caption{Inputs to nsq, number of squares of inductor}
\begin{center}
\begin{tabular}{|c|c|}
\hline
Input & Confidence\\\hlineB{2}
wI, width of inductor & 100\%\\\hline
lI, length of inductor & 100\%\\\hline
\end{tabular}
\end{center}
%\label{tab:VI}
\end{table}

\begin{align*}
n_\text{sq} &= \frac{w_I}{l_I}
\end{align*}

\subsection{nu\_opt, frequency of photons}

\begin{table}[H]
\caption{Inputs to nu\_opt, frequency of photons}
\begin{center}
\begin{tabular}{|c|c|}
\hline
Input & Confidence\\\hlineB{2}
lamb, wavelength of photons & 100\%\\\hline
\end{tabular}
\end{center}
%\label{tab:VI}
\end{table}

\begin{align*}
\nu_\text{opt} &= c\lambda
\end{align*}

\subsection{E\_gamma, energy of photons}

\begin{table}[H]
\caption{Inputs to E\_gamma, energy of photons}
\begin{center}
\begin{tabular}{|c|c|}
\hline
Input & Confidence\\\hlineB{2}
nu\_opt, frequency of photons & 100\%\\\hline
\end{tabular}
\end{center}
%\label{tab:VI}
\end{table}

\begin{align*}
E_\gamma &= h\nu_\text{opt}
\end{align*}

\subsection{delta0, gap energy at 0K}

\begin{table}[H]
Flanigan p.20 and Jay
\caption{Inputs to delta0, gap energy at 0K}
\begin{center}
\begin{tabular}{|c|c|}
\hline
Input & Confidence\\\hlineB{2}
Tc, critical temperature & 100\%\\\hline
\end{tabular}
\end{center}
%\label{tab:VI}
\end{table}

\begin{align*}
\Delta_0 &= 1.764kT_c
\end{align*}

\subsection{delta, gap energy at T}
Zmuidzinas eq.4, basically identical to delta0.
\begin{table}[H]
\caption{Inputs to delta, gap energy at T}
\begin{center}
\begin{tabular}{|c|c|}
\hline
Input & Confidence\\\hlineB{2}
delta0, gap energy at 0K & 100\%\\\hline
\end{tabular}
\end{center}
%\label{tab:VI}
\end{table}

\begin{align*}
\Delta &= \Delta_0\left(1 -\sqrt{2\pi k\Delta_0}e^{\frac{\Delta_0}{kT}}\right)
\end{align*}

\subsection{sigma\_n, normal conductivity just above Tc}
Flanigan (private correspondence)
\begin{table}[H]
\caption{Inputs to sigma\_n, normal conductivity just above Tc}
\begin{center}
\begin{tabular}{|c|c|}
\hline
Input & Confidence\\\hlineB{2}
R\_nsp, normal sheet resistivity & 100\%\\\hline
tI, thickness of inductor & 100\%\\\hline
\end{tabular}
\end{center}
%\label{tab:VI}
\end{table}

\begin{align*}
\sigma_n &= \frac{1}{R_\text{nsq} t_I}
\end{align*}

\subsection{lambL, (London) penetration depth in the thin film limit}
Zmuidzinas eq.12 (currently unused)
\begin{table}[H]
\caption{Inputs to lambL, (London) penetration depth in the thin film limit}
\begin{center}
\begin{tabular}{|c|c|}
\hline
Input & Confidence\\\hlineB{2}
sigma\_n, normal conductivity just above Tc & 50\%\\\hline
tI, thickness of inductor & 100\%\\\hline
\end{tabular}
\end{center}
%\label{tab:VI}
\end{table}

\begin{align*}
\lambda_L &= \frac{\hbar}{\pi\mu_0 t_I \Delta \sigma_n}
\end{align*}

\subsection{N0, single-spin density of electron states at Fermi energy}
Flanigan eq.3.10
\begin{table}[H]
\caption{Inputs to N0, single-spin density of electron states at Fermi energy}
\begin{center}
\begin{tabular}{|c|c|}
\hline
Input & Confidence\\\hlineB{2}
T, operating temperature & 100\%\\\hline
delta0, gap energy at 0K & 100\%\\\hline
\end{tabular}
\end{center}
%\label{tab:VI}
\end{table}

\begin{align*}
N_0 &= \frac{n_\text{qp,0}}{2\sqrt{2\pi kT \Delta_0} e^{-\frac{\Delta_0}{kT}}}
\end{align*}

\subsection{N\_qp\_photon, number of quasiparticles produced per photon}
Flanigan eq.3.67
\begin{table}[H]
\caption{Inputs to N\_qp\_photon, number of quasiparticles produced per photon}
\begin{center}
\begin{tabular}{|c|c|}
\hline
Input & Confidence\\\hlineB{2}
eta\_pb, pair-breaking efficiency & 100\%\\\hline
nu\_opt, frequency of photons & 100\%\\\hline
delta, gap energy at T & 100\%\\\hline
\end{tabular}
\end{center}
%\label{tab:VI}
\end{table}

\begin{align*}
N_\text{qp,phot} &= \frac{h \eta_\text{pb}\nu_\text{opt}}{\Delta}
\end{align*}

\section{Intrinsic parameters (intrinsic to material)}
\subsection{R\_qp, intrinsic quasiparticle recombination constant}
Flanigan eq.3.14
\begin{table}[H]
\caption{Inputs to R\_qp, intrinsic quasiparticle recombination constant}
\begin{center}
\begin{tabular}{|c|c|}
\hline
Input & Confidence\\\hlineB{2}
delta0, gap energy at 0K & 100\%\\\hline
N0, single-spin density of electron states at Fermi energy & 90\%\\\hline
tau0, characteristic electron-phonon interaction time & 0\%\\\hline
\end{tabular}
\end{center}
%\label{tab:VI}
\end{table}

\begin{align*}
R_\text{qp} &= \frac{2\left(\frac{\Delta_0}{kT_c}\right)^3}{N_0 \Delta_0 \tau_0}
\end{align*}

\section{Thermal parameters (depend on T)}
\subsection{n\_qp\_therm, quasiparticle density due to thermal effects at T}
Flanigan eq.3.10
\begin{table}[H]
\caption{Inputs to n\_qp\_therm, quasiparticle density due to thermal effects at T}
\begin{center}
\begin{tabular}{|c|c|}
\hline
Input & Confidence\\\hlineB{2}
delta0, gap energy at 0K & 100\%\\\hline
N0, single-spin density of electron states at Fermi energy & 90\%\\\hline
T, operating temperature & 100\%\\\hline
\end{tabular}
\end{center}
%\label{tab:VI}
\end{table}

\begin{align*}
n_\text{qp,therm} &= 2 N_0 \sqrt{2\pi kT\Delta_0} e^{-\frac{\Delta_0}{kT}}
\end{align*}

\subsection{gamma\_G, low-temperature thermal generation rate at T}
Flanigan eq.3.17
\begin{table}[H]
\caption{Inputs to gamma\_G, low-temperature thermal generation rate at T}
\begin{center}
\begin{tabular}{|c|c|}
\hline
Input & Confidence\\\hlineB{2}
delta0, gap energy at 0K & 100\%\\\hline
N0, single-spin density of electron states at Fermi energy & 90\%\\\hline
tau0, characteristic electron-phonon interaction time & 0\%\\\hline
T, operating temperature & 100\%\\\hline
Tc, critical temperature & 100\%\\\hline
\end{tabular}
\end{center}
%\label{tab:VI}
\end{table}

\begin{align*}
\gamma_G &= \frac{16 N_0 \Delta_0^3 \pi T}{\tau_0 k^2 T_c^3} e^{-\frac{2\Delta_0}{kT}}
\end{align*}

\section{Steady-state parameters}
\subsection{tau\_phon\_es, phonon escape time}
Flanigan eq.3.19 (currently unused)
\begin{table}[H]
\caption{Inputs to tau\_phon\_es, phonon escape time}
\begin{center}
\begin{tabular}{|c|c|}
\hline
Input & Confidence\\\hlineB{2}
tI, thickness of inductor & 100\%\\\hline
eta\_phon\_trans, transmission probability per encounter & 0\%\\\hline
s, probably speed of sound & 0\%\\\hline
\end{tabular}
\end{center}
%\label{tab:VI}
\end{table}

\begin{align*}
\tau_\text{phon,es} &= \frac{4 t_I}{s\eta_\text{phon, es}}\\
\eta_\text{phon,es} &= \num{1e-9}\\
s &= \SI{6.4e3}{ms^{-1}}
\end{align*}

\subsection{F\_phon, phonon trapping factor}
Flanigan eq.3.20 (currently unused)
\begin{table}[H]
\caption{Inputs to F\_phon, phonon trapping factor}
\begin{center}
\begin{tabular}{|c|c|}
\hline
Input & Confidence\\\hlineB{2}
tau\_phon\_br, time for phonon to break Cooper pair & 0\%\\\hline
tau\_phon\_es, phonon escape time & 0\%\\\hline
\end{tabular}
\end{center}
%\label{tab:VI}
\end{table}

\begin{align*}
F_\text{phon} &= 1 +\frac{\tau_\text{phon,es}}{\tau_\text{phon,br}}
\end{align*}

\subsection{R\_eff, effective quasiparticle recombination constant}
Flanigan p.31 (currently ignores F\_phon)
\begin{table}[H]
\caption{Inputs to R\_eff, effective quasiparticle recombination constant}
\begin{center}
\begin{tabular}{|c|c|}
\hline
Input & Confidence\\\hlineB{2}
R\_qp, intrinsic quasiparticle recombination constant & 0\%\\\hline
F\_phon, phonon trapping factor & 0\%\\\hline
\end{tabular}
\end{center}
%\label{tab:VI}
\end{table}

\begin{align*}
R_\text{eff} &= \frac{R_\text{qp}}{F_\text{phon}}
\end{align*}

\subsection{tau\_qp, quasiparticle relaxation time}
Flanigan p.46 (currently unused)
\begin{table}[H]
\caption{Inputs to tau\_qp, quasiparticle relaxation time}
\begin{center}
\begin{tabular}{|c|c|}
\hline
Input & Confidence\\\hlineB{2}
R\_eff, effective quasiparticle recombination constant & 0\%\\\hline
gamma\_G, low-temperature thermal generation rate at T & 0\%\\\hline
\end{tabular}
\end{center}
%\label{tab:VI}
\end{table}

\begin{align*}
\tau_\text{qp} &= \frac{1}{\sqrt{4R_\text{eff}\gamma_G}}
\end{align*}

\subsection{n\_qp\_ss, steady-state quasiparticle density}
Flanigan p.43. Should match nqp0, assuming ignoring F\_phon. tau0 term cancels out from R\_eff and gamma\_G.
\begin{table}[H]
\caption{Inputs to n\_qp\_ss, steady-state quasiparticle density}
\begin{center}
\begin{tabular}{|c|c|}
\hline
Input & Confidence\\\hlineB{2}
R\_eff, effective quasiparticle recombination constant & 0\%\\\hline
gamma\_G, low-temperature thermal generation rate at T & 0\%\\\hline
\end{tabular}
\end{center}
%\label{tab:VI}
\end{table}

\begin{align*}
n_\text{qp,ss} &= \sqrt{\frac{\gamma_G}{R_\text{eff}}}
\end{align*}

\subsection{N\_qp\_ss, steady-state quasiparticle number in resonator}
Flanigan p.57. tau0 term cancels out from R\_eff and gamma\_G.
\begin{table}[H]
\caption{Inputs to N\_qp\_ss, steady-state quasiparticle number in resonator}
\begin{center}
\begin{tabular}{|c|c|}
\hline
Input & Confidence\\\hlineB{2}
R\_eff, effective quasiparticle recombination constant & 0\%\\\hline
gamma\_G, low-temperature thermal generation rate at T & 0\%\\\hline
VI, volume of inductor & 100\%\\\hline
\end{tabular}
\end{center}
%\label{tab:VI}
\end{table}

\begin{align*}
N_\text{qp,ss} &= \sqrt{\frac{V_I\gamma_G}{R_\text{eff}}}
\end{align*}

\section{Optical generation}
\subsection{gamma\_opt, constant optical power quasiparticle generation rate}
Flanigan eq.3.68
\begin{table}[H]
\caption{Inputs to gamma\_opt, constant optical power quasiparticle generation rate}
\begin{center}
\begin{tabular}{|c|c|}
\hline
Input & Confidence\\\hlineB{2}
P\_opt, incident optical power & 100\%\\\hline
dGamma\_dP, responsivity of quasiparticle generation rate to optical power & 100\%\\\hline
eta\_pb, pair-breaking efficiency & 100\%\\\hline
delta, gap energy at T & 100\%\\\hline
\end{tabular}
\end{center}
%\label{tab:VI}
\end{table}

\begin{align*}
\gamma_\text{opt} &= \frac{dP_\text{abs}}{dP_\text{inc}}\frac{P_\text{opt}\eta_\text{pb}}{\Delta}
\end{align*}

\section{All together now}
\subsection{N\_qp\_tot, total number of quasiparticles in resonator due to thermal and constant optical power effects}
Flanigan p.57. tau0 term cancels out between R\_eff and gamma\_G, but not between R\_eff and gamma\_opt. Effects are transmitted forward to fnew and Q\_qp.
\begin{table}[H]
\caption{Inputs to N\_qp\_tot, total number of quasiparticles in resonator due to thermal and constant optical power effects}
\begin{center}
\begin{tabular}{|c|c|}
\hline
Input & Confidence\\\hlineB{2}
P\_opt, incident optical power & 100\%\\\hline
gamma\_opt, constant optical power quasiparticle generation rate & 100\%\\\hline
R\_eff, effective quasiparticle recombination constant & 0\%\\\hline
gamma\_G, low-temperature thermal generation rate at T & 0\%\\\hline
VI, volume of inductor & 100\%\\\hline
\end{tabular}
\end{center}
%\label{tab:VI}
\end{table}

\begin{align*}
N_\text{qp,tot} &= \sqrt{\frac{V_I(\gamma_G +\gamma_\text{opt}(P_\text{opt}))}{R_\text{eff}}}
\end{align*}

\section{Complex conductivity}
\subsection{sigma1\_0, real part of complex conductivity at T=0K}
Flanigan p.35
\begin{table}[H]
\caption{Inputs to sigma1\_0, real part of complex conductivity at T=0K}
\begin{center}
\begin{tabular}{|c|c|}
\hline
Input & Confidence\\\hlineB{2}
\end{tabular}
\end{center}
%\label{tab:VI}
\end{table}

\begin{align*}
\sigma_1(0) &= 0
\end{align*}

\subsection{sigma2\_0, imag part of complex conductivity at T=0K}
Flanigan p.35
\begin{table}[H]
\caption{Inputs to sigma2\_0, imag part of complex conductivity at T=0K}
\begin{center}
\begin{tabular}{|c|c|}
\hline
Input & Confidence\\\hlineB{2}
sigma\_n, normal conductivity just above Tc & 100\%\\\hline
delta0, gap energy at 0K & 100\%\\\hline
f, readout frequency & 100\%\\\hline
\end{tabular}
\end{center}
%\label{tab:VI}
\end{table}

\begin{align*}
\sigma_2(0) &= \frac{\pi\Delta_0\sigma_n}{hf}
\end{align*}

\subsection{sigma1rat, ratio of real part of complex conductivity to quasiparticle density response at T}
Flanigan eq.3.79. Used in dsig1\_dN.
\begin{table}[H]
\caption{Inputs to sigma1rat, ratio of real part of complex conductivity to quasiparticle density response at T}
\begin{center}
\begin{tabular}{|c|c|}
\hline
Input & Confidence\\\hlineB{2}
delta0, gap energy at 0K & 100\%\\\hline
nu\_opt, frequency of photons & 100\%\\\hline
T, operating temperature & 100\%\\\hline
\end{tabular}
\end{center}
%\label{tab:VI}
\end{table}

\begin{align*}
\Upsilon_{\sigma_1} &= \sqrt{\frac{8\Delta_0}{\pi^3 kT}}\sinh\frac{h\nu_\text{opt}}{2kT} K_0\left(\frac{h\nu_\text{opt}}{2kT}\right)
\end{align*}

\subsection{sigma2rat, ratio of imag part of complex conductivity to quasiparticle density response at T}
Flanigan eq.3.80. Used in dsig2\_dN.
\begin{table}[H]
\caption{Inputs to sigma2rat, ratio of imag part of complex conductivity to quasiparticle density response at T}
\begin{center}
\begin{tabular}{|c|c|}
\hline
Input & Confidence\\\hlineB{2}
delta0, gap energy at 0K & 100\%\\\hline
nu\_opt, frequency of photons & 100\%\\\hline
T, operating temperature & 100\%\\\hline
\end{tabular}
\end{center}
%\label{tab:VI}
\end{table}

\begin{align*}
\Upsilon_{\sigma_2} &= -1 -\sqrt{\frac{2\Delta_0}{\pi kT}}e^{-\frac{h\nu_\text{opt}}{2kT}} I_0\left(\frac{h\nu_\text{opt}}{2kT}\right)
\end{align*}

\subsection{sigma1, real part of complex conductivity at T}
Uses dsig1\_dN.
\begin{table}[H]
\caption{Inputs to sigma1, real part of complex conductivity at T}
\begin{center}
\begin{tabular}{|c|c|}
\hline
Input & Confidence\\\hlineB{2}
P\_opt, incident optical power & 100\%\\\hline
N\_qp\_tot, total number of quasiparticles in resonator & 90\%\\\hline
sigma1\_0, real part of complex conductivity at T=0K & 100\%\\\hline
dsig1\_dN, responsivity of sigma1 against N\_qp\_tot & 100\%\\\hline
f, readout frequency & 100\%\\\hline
\end{tabular}
\end{center}
%\label{tab:VI}
\end{table}

\begin{align*}
\sigma_1(f) &= N_\text{qp,tot}(P_\text{opt})\frac{d\sigma_1(f)}{dN_\text{qp}} +\sigma_1(0)
\end{align*}

\subsection{sigma2, imag part of complex conductivity at T}
Uses dsig2\_dN.
\begin{table}[H]
\caption{Inputs to sigma2, imag part of complex conductivity at T}
\begin{center}
\begin{tabular}{|c|c|}
\hline
Input & Confidence\\\hlineB{2}
P\_opt, incident optical power & 100\%\\\hline
N\_qp\_tot, total number of quasiparticles in resonator & 90\%\\\hline
sigma2\_0, imag part of complex conductivity at T=0K & 100\%\\\hline
dsig2\_dN, responsivity of sigma2 against N\_qp\_tot & 100\%\\\hline
f, readout frequency & 100\%\\\hline
\end{tabular}
\end{center}
%\label{tab:VI}
\end{table}

\begin{align*}
\sigma_2(f) &= N_\text{qp,tot}(P_\text{opt})\frac{d\sigma_2(f)}{dN_\text{qp}} +\sigma_2(f,0)
\end{align*}

\subsection{sigma, complex conductivity at T}
Flanigan p.34
\begin{table}[H]
\caption{Inputs to sigma, complex conductivity at T}
\begin{center}
\begin{tabular}{|c|c|}
\hline
Input & Confidence\\\hlineB{2}
P\_opt, incident optical power & 100\%\\\hline
sigma1, real part of complex conductivity at T & 90\%\\\hline
sigma2, imag part of complex conductivity at T & 90\%\\\hline
f, readout frequency & 100\%\\\hline
\end{tabular}
\end{center}
%\label{tab:VI}
\end{table}

\begin{align*}
\sigma(f) &= \sigma_1(f,P_\text{opt}) -i\sigma_2(f,P_\text{opt})
\end{align*}

\section{Surface impedance, reactance, (kinetic) inductance, resistance}
\subsection{Zs\_0, surface impedance in thin film local limit at T=0K}
Flanigan eq.3.35
\begin{table}[H]
\caption{Inputs to Zs\_0, surface impedance in thin film local limit at T=0K}
\begin{center}
\begin{tabular}{|c|c|}
\hline
Input & Confidence\\\hlineB{2}
sigma2\_0, imag part of complex conductivity at T=0K & 100\%\\\hline
tI, thickness of inductor & 100\%\\\hline
f, readout frequency & 100\%\\\hline
\end{tabular}
\end{center}
%\label{tab:VI}
\end{table}

\begin{align*}
Z_s(f,0) &= \frac{i}{t_I\sigma_2(f,0)}
\end{align*}

\subsection{Zs, surface impedance in thin film local limit at T}
Flanigan eq.3.35
\begin{table}[H]
\caption{Inputs to Zs, surface impedance in thin film local limit at T}
\begin{center}
\begin{tabular}{|c|c|}
\hline
Input & Confidence\\\hlineB{2}
P\_opt, incident optical power & 100\%\\\hline
sigma, complex conductivity at T & 90\%\\\hline
tI, thickness of inductor & 100\%\\\hline
f, readout frequency & 100\%\\\hline
\end{tabular}
\end{center}
%\label{tab:VI}
\end{table}

\begin{align*}
Z_s(f) &= \frac{1}{t_I\sigma(f,P_\text{opt})}
\end{align*}

\subsection{Xs\_0, surface reactance in thin film local limit at T=0K}
Flanigan p.36
\begin{table}[H]
\caption{Inputs to Xs\_0, surface reactance in thin film local limit at T=0K}
\begin{center}
\begin{tabular}{|c|c|}
\hline
Input & Confidence\\\hlineB{2}
Zs\_0, surface impedance in thin film local limit at T=0K & 100\%\\\hline
f, readout frequency & 100\%\\\hline
\end{tabular}
\end{center}
%\label{tab:VI}
\end{table}

\begin{align*}
X_s(f,0) &= \Im(Z_s(f,0))
\end{align*}

\subsection{Xs, surface reactance in thin film local limit at T}
Flanigan p.36
\begin{table}[H]
\caption{Inputs to Xs, surface reactance in thin film local limit at T}
\begin{center}
\begin{tabular}{|c|c|}
\hline
Input & Confidence\\\hlineB{2}
P\_opt, incident optical power & 100\%\\\hline
Zs, surface impedance in thin film local limit at T & 90\%\\\hline
f, readout frequency & 100\%\\\hline
\end{tabular}
\end{center}
%\label{tab:VI}
\end{table}

\begin{align*}
X_s(f) &= \Im(Z_s(f,P_\text{opt}))
\end{align*}

\subsection{Rs, surface resistance in thin film local limit at T}
Flanigan p.36
\begin{table}[H]
\caption{Inputs to Rs, surface resistance in thin film local limit at T}
\begin{center}
\begin{tabular}{|c|c|}
\hline
Input & Confidence\\\hlineB{2}
P\_opt, incident optical power & 100\%\\\hline
Zs, surface impedance in thin film local limit at T & 90\%\\\hline
f, readout frequency & 100\%\\\hline
\end{tabular}
\end{center}
%\label{tab:VI}
\end{table}

\begin{align*}
R_s(f) &= \Re(Z_s(f,P_\text{opt}))
\end{align*}

\subsection{Lk\_0, kinetic inductance in thin film local limit at T=0K}
Flanigan p.36
\begin{table}[H]
\caption{Inputs to Lk\_0, kinetic inductance in thin film local limit at T=0K}
\begin{center}
\begin{tabular}{|c|c|}
\hline
Input & Confidence\\\hlineB{2}
nsq, number of squares of inductor & 100\%\\\hline
tI, thickness of inductor & 100\%\\\hline
delta0, gap energy at 0K & 100\%\\\hline
sigma\_n, normal conductivity just above Tc & 100\%\\\hline
\end{tabular}
\end{center}
%\label{tab:VI}
\end{table}

\begin{align*}
L_k(0) &= \frac{n_\text{sq}h}{2\pi^2 t_I\Delta_0\sigma_n}
\end{align*}

\subsection{Lk, kinetic inductance in thin film local limit at T}
Flanigan p.36
\begin{table}[H]
\caption{Inputs to Lk, kinetic inductance in thin film local limit at T}
\begin{center}
\begin{tabular}{|c|c|}
\hline
Input & Confidence\\\hlineB{2}
P\_opt, incident optical power & 100\%\\\hline
tI, thickness of inductor & 100\%\\\hline
delta0, gap energy at 0K & 100\%\\\hline
sigma\_n, normal conductivity just above Tc & 100\%\\\hline
N0, single-spin density of electron states at Fermi energy & 90\%\\\hline
VI, volume of inductor & 100\%\\\hline
\end{tabular}
\end{center}
%\label{tab:VI}
\end{table}

\begin{align*}
L_k &= L_k(0)\left(1 -\frac{N_\text{qp,tot}(P_\text{opt})\Upsilon_{\sigma_2}}{2N_0\Delta_0 V_I +N_\text{qp,tot}(P_\text{opt})\Upsilon_{\sigma_2}}\right)
\end{align*}

\section{Resonant frequency}
\subsection{alpha, effective kinetic inductance fraction in thin film local limit}
Flanigan eq.3.62
\begin{table}[H]
\caption{Inputs to alpha, effective kinetic inductance fraction in thin film local limit}
\begin{center}
\begin{tabular}{|c|c|}
\hline
Input & Confidence\\\hlineB{2}
Lk\_0, kinetic inductance in thin film local limit at T=0K & 100\%\\\hline
Lg, geometric inductance & 100\%\\\hline
\end{tabular}
\end{center}
%\label{tab:VI}
\end{table}

\begin{align*}
\alpha &= \frac{L_k(0)}{L_g +L_k(0)}
\end{align*}

\subsection{f0, resonant frequency of resonator circuit at T=0K}
Flanigan p.52
\begin{table}[H]
\caption{Inputs to f0, resonant frequency of resonator circuit at T=0K}
\begin{center}
\begin{tabular}{|c|c|}
\hline
Input & Confidence\\\hlineB{2}
Lk\_0, kinetic inductance in thin film local limit at T=0K & 100\%\\\hline
Lg, geometric inductance & 100\%\\\hline
C, capacitor capacitance in circuit & 100\%\\\hline
\end{tabular}
\end{center}
%\label{tab:VI}
\end{table}

\begin{align*}
f_0 &= \frac{1}{2\pi \sqrt{C(L_g +L_k(0))}}
\end{align*}

\subsection{ffrac, fractional frequency shift in resonant frequency of circuit}
Flanigan eq.3.63
\begin{table}[H]
\caption{Inputs to ffrac, fractional frequency shift in resonant frequency of circuit}
\begin{center}
\begin{tabular}{|c|c|}
\hline
Input & Confidence\\\hlineB{2}
alpha, effective kinetic inductance fraction in thin film local limit & 100\%\\\hline
P\_opt, incident optical power & 100\%\\\hline
Lk, kinetic inductance in thin film local limit at T & 90\%\\\hline
Lk\_0, kinetic inductance in thin film local limit at T=0K & 100\%\\\hline
\end{tabular}
\end{center}
%\label{tab:VI}
\end{table}

\begin{align*}
s &= \frac{\alpha}{2}\frac{L_k(P_\text{opt}) -L_k(0)}{L_k(0)}
\end{align*}

\subsection{fnew, resonant frequency of resonator circuit in thin film local limit}
Flanigan eq.3.61
\begin{table}[H]
\caption{Inputs to fnew, resonant frequency of resonator circuit in thin film local limit}
\begin{center}
\begin{tabular}{|c|c|}
\hline
Input & Confidence\\\hlineB{2}
P\_opt, incident optical power & 100\%\\\hline
f0, resonant frequency of resonator circuit at T=0K & 100\%\\\hline
ffrac, fractional frequency shift in resonant frequency of circuit & 90\%\\\hline
\end{tabular}
\end{center}
%\label{tab:VI}
\end{table}

\begin{align*}
f_\text{new} &= f_0(1 -s)
\end{align*}

\subsection{fdet, detuning of resonant frequency from readout frequency}
Flanigan p.50
\begin{table}[H]
\caption{Inputs to fdet, detuning of resonant frequency from readout frequency}
\begin{center}
\begin{tabular}{|c|c|}
\hline
Input & Confidence\\\hlineB{2}
P\_opt, incident optical power & 100\%\\\hline
f, readout frequency & 100\%\\\hline
fnew, resonant frequency of resonator circuit in thin film local limit & 90\%\\\hline
\end{tabular}
\end{center}
%\label{tab:VI}
\end{table}

\begin{align*}
x &= \frac{f}{f_\text{new}} -1
\end{align*}

\section{Quality factors}
\subsection{Q\_qp, quality factor of resonator circuit from quasiparticles}
Flanigan eq.3.64
\begin{table}[H]
\caption{Inputs to Q\_qp, quality factor of resonator circuit from quasiparticles}
\begin{center}
\begin{tabular}{|c|c|}
\hline
Input & Confidence\\\hlineB{2}
alpha, effective kinetic inductance fraction in thin film local limit & 100\%\\\hline
P\_opt, incident optical power & 100\%\\\hline
Xs\_0, surface reactance in thin film local limit at T=0K & 100\%\\\hline
Rs, surface resistance in thin film local limit at T & 90\%\\\hline
f, readout frequency & 100\%\\\hline
\end{tabular}
\end{center}
%\label{tab:VI}
\end{table}

\begin{align*}
Q_\text{qp}(f) &= \frac{X_s(f,0)}{\alpha R_s(f,P_\text{opt})}
\end{align*}

\subsection{Qr, quality factor of resonator circuit in thin film local limit}
Flanigan eq.3.58. Assumes internal quality factor Q\_i is dominated by Q\_qp.
\begin{table}[H]
\caption{Inputs to Qr, quality factor of resonator circuit in thin film local limit}
\begin{center}
\begin{tabular}{|c|c|}
\hline
Input & Confidence\\\hlineB{2}
P\_opt, incident optical power & 100\%\\\hline
Q\_qp, quality factor of resonator circuit from quasiparticles & 90\%\\\hline
Qc, coupling quality factor & 100\%\\\hline
f, readout frequency & 100\%\\\hline
\end{tabular}
\end{center}
%\label{tab:VI}
\end{table}

\begin{align*}
Q_\text{r} &= \left(\frac{1}{Q_c} +\frac{1}{Q_\text{qp}(f,P_\text{opt})}\right)^{-1}
\end{align*}

\section{Responsivities}
\subsection{dPabs\_dPinc, responsivity of absorbed optical power to incident optical power}
Flanigan eq.3.66
\begin{table}[H]
\caption{Inputs to dPabs\_dPinc, responsivity of absorbed optical power to incident optical power}
\begin{center}
\begin{tabular}{|c|c|}
\hline
Input & Confidence\\\hlineB{2}
eta\_opt, optical efficiency & 100\%\\\hline
\end{tabular}
\end{center}
%\label{tab:VI}
\end{table}

\begin{align*}
\frac{dP_\text{abs}}{dP_\text{inc}} &= \eta_\text{opt}
\end{align*}

\subsection{dGamma\_dP, responsivity of quasiparticle generation rate to optical power}
Flanigan eq.3.69
\begin{table}[H]
\caption{Inputs to dGamma\_dP, responsivity of quasiparticle generation rate to optical power}
\begin{center}
\begin{tabular}{|c|c|}
\hline
Input & Confidence\\\hlineB{2}
eta\_pb, pair-breaking efficiency & 100\%\\\hline
delta0, gap energy at 0K & 100\%\\\hline
\end{tabular}
\end{center}
%\label{tab:VI}
\end{table}

\begin{align*}
\frac{d\gamma_\text{opt}}{dP_\text{opt}} &= \frac{\eta_\text{pb}}{\Delta_0}
\end{align*}

\subsection{dN\_qp\_tot\_dGamma, responsivity of N\_qp\_tot to quasiparticle generation rate}
Flanigan eq.3.72
\begin{table}[H]
\caption{Inputs to dN\_qp\_tot\_dGamma, responsivity of N\_qp\_tot to quasiparticle generation rate}
\begin{center}
\begin{tabular}{|c|c|}
\hline
Input & Confidence\\\hlineB{2}
P\_opt, incident optical power & 100\%\\\hline
gamma\_opt, constant optical power quasiparticle generation rate & 100\%\\\hline
R\_eff, effective quasiparticle recombination constant & 0\%\\\hline
gamma\_G, low-temperature thermal generation rate at T & 0\%\\\hline
VI, volume of inductor & 100\%\\\hline
\end{tabular}
\end{center}
%\label{tab:VI}
\end{table}

\begin{align*}
\frac{dN_\text{qp,tot}}{d\gamma} &= \frac{1}{2}\sqrt{\frac{V_I}{R_\text{eff}(\gamma_G +\gamma_\text{opt}(P_\text{opt}))}}
\end{align*}

\subsection{dsig1\_dN, responsivity of sigma1 to N\_qp\_tot}
Flanigan eq.3.81. Uses sigma1rat and used in calculation of sigma1.
\begin{table}[H]
\caption{Inputs to dsig1\_dN, responsivity of sigma1 to N\_qp\_tot}
\begin{center}
\begin{tabular}{|p{13.5cm}|p{2cm}|}
\hline
Input & Confidence\\\hlineB{2}
sigma2\_0, imag part of complex conductivity at T=0K & 100\%\\\hline
sigma1rat, ratio of real part of complex conductivity to quasiparticle density response at T & 100\%\\\hline
N0, single-spin density of electron states at Fermi energy & 100\%\\\hline
delta0, gap energy at 0K & 100\%\\\hline
VI, volume of inductor & 100\%\\\hline
f, readout frequency & 100\%\\\hline
\end{tabular}
\end{center}
%\label{tab:VI}
\end{table}

\begin{align*}
\frac{d\sigma_1(f)}{dN_\text{qp,tot}} &= \frac{\sigma_2(f,0)\Upsilon_{\sigma_1}}{2N_0 \Delta_0 V_I}
\end{align*}

\subsection{dsig2\_dN, responsivity of sigma2 to N\_qp\_tot}
Flanigan eq.3.82. Uses sigma2rat and used in calculation of sigma2.
\begin{table}[H]
\caption{Inputs to dsig2\_dN, responsivity of sigma2 to N\_qp\_tot}
\begin{center}
\begin{tabular}{|p{13.5cm}|p{2cm}|}
\hline
Input & Confidence\\\hlineB{2}
sigma2\_0, imag part of complex conductivity at T=0K & 100\%\\\hline
sigma2rat, ratio of real part of complex conductivity to quasiparticle density response at T & 100\%\\\hline
N0, single-spin density of electron states at Fermi energy & 100\%\\\hline
delta0, gap energy at 0K & 100\%\\\hline
VI, volume of inductor & 100\%\\\hline
f, readout frequency & 100\%\\\hline
\end{tabular}
\end{center}
%\label{tab:VI}
\end{table}

\begin{align*}
\frac{d\sigma_2(f)}{dN_\text{qp,tot}} &= \frac{\sigma_2(f,0)\Upsilon_{\sigma_2}}{2N_0 \Delta_0 V_I}
\end{align*}

\subsection{dRs\_dsig1, responsivity of surface resistance Rs to sigma1}
Flanigan eq.3.83
\begin{table}[H]
\caption{Inputs to dRs\_dsig1, responsivity of surface resistance Rs to sigma1}
\begin{center}
\begin{tabular}{|c|c|}
\hline
Input & Confidence\\\hlineB{2}
Xs\_0, surface reactance in thin film local limit at T=0K & 100\%\\\hline
sigma2\_0, imag part of complex conductivity at T=0K & 100\%\\\hline
f, readout frequency & 100\%\\\hline
\end{tabular}
\end{center}
%\label{tab:VI}
\end{table}

\begin{align*}
\frac{dR_s(f)}{d\sigma_1} &= \frac{X_s(f,0)}{\sigma_2(f,0)}
\end{align*}

\subsection{dXs\_dsig2, responsivity of surface reactance Xs to sigma2}
Flanigan eq.3.84
\begin{table}[H]
\caption{Inputs to dXs\_dsig2, responsivity of surface reactance Xs to sigma2}
\begin{center}
\begin{tabular}{|c|c|}
\hline
Input & Confidence\\\hlineB{2}
Xs\_0, surface reactance in thin film local limit at T=0K & 100\%\\\hline
sigma2\_0, imag part of complex conductivity at T=0K & 100\%\\\hline
f, readout frequency & 100\%\\\hline
\end{tabular}
\end{center}
%\label{tab:VI}
\end{table}

\begin{align*}
\frac{dX_s(f)}{d\sigma_2} &= -\frac{X_s(f,0)}{\sigma_2(f,0)}
\end{align*}

\subsection{dlambqp\_dRs, responsivity of quasiparticle loss factor lambda\_qp to surface resistance Rs}
Flanigan eq.3.87
\begin{table}[H]
\caption{Inputs to dlambqp\_dRs, responsivity of quasiparticle loss factor lambda\_qp to surface resistance Rs}
\begin{center}
\begin{tabular}{|c|c|}
\hline
Input & Confidence\\\hlineB{2}
alpha, effective kinetic inductance fraction in thin film local limit & 100\%\\\hline
P\_opt, incident optical power & 100\%\\\hline
Xs\_0, surface reactance in thin film local limit at T=0K & 100\%\\\hline
f, readout frequency & 100\%\\\hline
\end{tabular}
\end{center}
%\label{tab:VI}
\end{table}

\begin{align*}
\frac{d\lambda_\text{qp}}{dR_s(f)} &= \frac{\alpha(P_\text{opt})}{X_s(f,0)}
\end{align*}

\subsection{dx\_dXs, responsivity of frequency detuning x to surface reactance Xs}
Flanigan eq.3.88
\begin{table}[H]
\caption{Inputs to dx\_dXs, responsivity of frequency detuning x to surface reactance Xs}
\begin{center}
\begin{tabular}{|c|c|}
\hline
Input & Confidence\\\hlineB{2}
alpha, effective kinetic inductance fraction in thin film local limit & 100\%\\\hline
P\_opt, incident optical power & 100\%\\\hline
Xs\_0, surface reactance in thin film local limit at T=0K & 100\%\\\hline
f, readout frequency & 100\%\\\hline
\end{tabular}
\end{center}
%\label{tab:VI}
\end{table}

\begin{align*}
\frac{dx}{dX_s(f)} &= \frac{\alpha(P_\text{opt})}{2X_s(f,0)}
\end{align*}

\subsection{dQqp\_dRs, responsivity of quasiparticle quality factor Q\_qp to surface resistance Rs}

\begin{table}[H]
\caption{Inputs to dQqp\_dRs, responsivity of quasiparticle quality factor Q\_qp to surface resistance Rs}
\begin{center}
\begin{tabular}{|c|c|}
\hline
Input & Confidence\\\hlineB{2}
alpha, effective kinetic inductance fraction in thin film local limit & 100\%\\\hline
P\_opt, incident optical power & 100\%\\\hline
Q\_qp, quality factor of resonator circuit from quasiparticles & 100\%\\\hline
Xs\_0, surface reactance in thin film local limit at T=0K & 100\%\\\hline
f, readout frequency & 100\%\\\hline
\end{tabular}
\end{center}
%\label{tab:VI}
\end{table}

\begin{align*}
\frac{dQ_\text{qp}(f)}{dR_s(f)} &= -\frac{Q_\text{qp}(f)^2 \alpha(P_\text{opt})}{X_s(f,0)}
\end{align*}

\section{S21}
\subsection{S21, resonator quality factor of resonator circuit in thin film local limit}
Flanigan eq.3.60
\begin{table}[H]
\caption{Inputs to S21, resonator quality factor of resonator circuit in thin film local limit}
\begin{center}
\begin{tabular}{|c|c|}
\hline
Input & Confidence\\\hlineB{2}
P\_opt, incident optical power & 100\%\\\hline
f, readout frequency & 100\%\\\hline
Q\_r, quality factor of resonator circuit in thin film local limit & 90\%\\\hline
Qc, coupling quality factor & 100\%\\\hline
fdet, detuning of resonant frequency from readout frequency & 90\%\\\hline
A, symmetry factor & 90\%\\\hline
\end{tabular}
\end{center}
%\label{tab:VI}
\end{table}

\begin{align*}
S_{21} &= 1 -\frac{Q_r(f,P_\text{opt})(1 +iA)}{Q_c(1 +2iQ_r(f,P_\text{opt})x(f,P_\text{opt}))}
\end{align*}

\section{NEP}
\subsection{nep\_phot, noise equivalent power (NEP) of photon noise}
Flanigan eq.5.5, from shot noise and wave noise
\begin{table}[H]
\caption{Inputs to nep\_phot, noise equivalent power (NEP) of photon noise}
\begin{center}
\begin{tabular}{|c|c|}
\hline
Input & Confidence\\\hlineB{2}
P\_opt, incident optical power & 100\%\\\hline
nu\_opt, frequency of photons & 100\%\\\hline
delta\_nuopt, optical bandwidth & 90\%\\\hline
\end{tabular}
\end{center}
%\label{tab:VI}
\end{table}

\begin{align*}
\text{NEP}(P_\text{opt}) &= \sqrt{2\left(h\nu P_\text{opt} +\frac{P_\text{opt}^2}{\Delta\nu}\right)}
\end{align*}

\subsection{nep\_rec, noise equivalent power (NEP) of recombination noise due to P\_opt}
Flanigan eq.5.19
\begin{table}[H]
\caption{Inputs to nep\_rec, noise equivalent power (NEP) of recombination noise due to P\_opt}
\begin{center}
\begin{tabular}{|c|c|}
\hline
Input & Confidence\\\hlineB{2}
P\_opt, incident optical power & 100\%\\\hline
delta0, gap energy at 0K & 100\%\\\hline
eta\_opt, optical efficiency & 100\%\\\hline
eta\_pb, pair-breaking efficiency & 100\%\\\hline
\end{tabular}
\end{center}
%\label{tab:VI}
\end{table}

\begin{align*}
\text{NEP}(P_\text{opt}) &= 2\sqrt{\frac{\Delta_0 P_\text{opt}}{\eta_\text{opt}\eta_\text{pb}}}
\end{align*}

\section{Finding P\_opt}
\subsection{P\_opt\_r, P\_opt value from fnew}

\begin{table}[H]
\caption{Inputs to P\_opt\_r, P\_opt value from fnew}
\begin{center}
\begin{tabular}{|c|c|}
\hline
Input & Confidence\\\hlineB{2}
fnew, resonant frequency of resonator circuit & 100\%\\\hline

\end{tabular}
\end{center}
%\label{tab:VI}
\end{table}

\begin{align*}
L_k(P_\text{opt}) &= 2(L_g +L_k(T=0))(1 -2\pi f_\text{new}\sqrt{C(L_g +L_k(T=0))}) +L_g\\
N_\text{qp,tot} &= -\frac{2N_0\Delta_0 V_I}{\Upsilon_{\sigma_2}}\left(1 +\frac{L_k(T=0)}{L_k(P_\text{opt})}\right)\\
P_\text{opt} &= \frac{\Delta}{\eta_\text{pb}\eta_\text{opt}}\left(\frac{N_\text{qp,tot}^2 R_\text{eff}}{V_I} -\gamma_G\right)
\end{align*}

\section{What doesn't work yet}
\subsection{f0 resonance frequency}
With parameters meant to have fnew(0)$=$500 MHz, i.e. nqp0$=$\num{e20} and R\_nsq$=$\SI{90}{\ohm}, and tau0 set at value for Al (\SI{4.38e-7}{s}), actual fnew(0)$=$1.281 GHz. Linked to R\_nsq.

\subsection{ffrac too low}
Fig.~\ref{fig:ffrac500}: With parameters set to have fnew(0)$=$500 kHz, i.e. nqp0$=$\num{e20} and R\_nsq$=$\SI{1.7536e4}{\ohm}, and tau0 set at value for Al (\SI{4.38e-7}{s}), ffrac changes very little with P\_opt. Linked to nqp0.

Fig.~\ref{fig:res10} and Fig.~\ref{fig:ffrac10}: With only change from default being tau0=\SI{10}{s}, ffrac follows empirical trend.

\begin{figure}[htbp]
\begin{center}
\includegraphics[width=0.8\textwidth]{ffrac500.png}
\caption{Fractional frequency shift with R\_nsq$=$\SI{1.7536e4}{\ohm} so fnew(0)$=$500 kHz.}
\label{fig:ffrac500}
\end{center}
\end{figure}

\begin{figure}[htbp]
\begin{center}
\includegraphics[width=0.8\textwidth]{res10.png}
\caption{Resonance curves with only tau0 changed from \SI{4.38e-7}{s} to \SI{10}{s}.}
\label{fig:res10}
\end{center}
\end{figure}

\begin{figure}[htbp]
\begin{center}
\includegraphics[width=0.8\textwidth]{ffrac10.png}
\caption{Fractional frequency shift with only tau0 changed from \SI{4.38e-7}{s} to \SI{10}{s}.}
\label{fig:ffrac10}
\end{center}
\end{figure}

\subsection{Qr too low or drops too quickly}
Fig.~\ref{fig:qr500}: With parameters set to have fnew(0)$=$500 kHz and tau0 set at value for Al (\SI{4.38e-7}{s}), Qr too low or drops too quickly with P\_opt. Linked to nqp0.

Fig.~\ref{fig:qr10}: With only change from default being tau0=\SI{10}{s}, Qr follows empirical trend.

\begin{figure}[htbp]
\begin{center}
\includegraphics[width=0.8\textwidth]{qr500.png}
\caption{Quality factor with R\_nsq$=$\SI{1.7536e4}{\ohm} so fnew(0)$=$500 kHz.}
\label{fig:qr500}
\end{center}
\end{figure}

\begin{figure}[htbp]
\begin{center}
\includegraphics[width=0.8\textwidth]{qr10.png}
\caption{Quality factor with only tau0 changed from \SI{4.38e-7}{s} to \SI{10}{s}.}
\label{fig:qr10}
\end{center}
\end{figure}

\newpage

\end{document}